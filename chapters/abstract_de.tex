\chapter*{Kurzfassung}

Schleifen sind ein unverzichtbares Konstrukt imperativer Programmiersprachen. Wie Turings frühe Arbeit zum Halteproblem zeigt, bringt die Ausdrucksstärke solcher Anweisungen allerdings unentscheidbare Probleme mit sich. Aus diesem Grund kann die Wirksamkeit eines automatischen Terminierungsbeweises nur anhand praktischer Beispiele demonstriert werden -- nach dem Satz von Rice gilt dies gleichermaßen für jede Form der automatischen Programmanalyse. Obwohl nach dem Stand der Technik einige, aber bei weitem nicht alle entscheidbaren Fälle bewältigt werden können, existiert noch keine etablierte Vorstellung von der Schwierigkeit Schleifen zu behandeln in der Scientific Community. Infolgedessen wurde auch das Auftreten "einfacher" und "schwieriger" Schleifen in der Praxis noch nicht beschrieben.

Wir empfehlen, dies in einer Studie zu untersuchen. Um Schleifen zu klassieren, bedienen wir uns einer neuen Forschungsrichtung in der Softwareverifikation: Anstatt Programme ausschließlich als mathematische Objekte zu betrachten, behandeln wir sie als von Menschen geschaffene Artefakte, die Informationen an die Leser des Programms richten. Diese sollten zum Zweck der automatischen Programmanalyse eingesetzt werden, sind aber meist nur informell (z.B.\ als Kommentar) oder implizit (z.B.\ als von Programmierern benutztes Muster) beschrieben. Daher benötigen wir eine Methode, derartige Informationen über Schleifen zu extrahieren.

Eine direkte Anwendung ist die empirische Evaluierung automatischer Programmanalyse -- wie beschrieben von zentraler Bedeutung zur Feststellung akademischer Leistungen als auch der Anwendbarkeit im Software Engineering. Die derzeit übliche Praxis erlaubt jedoch aufgrund der Vielzahl an verwendeten Benchmarks nur eingeschränkte Vergleichbarkeit. \emph{Benchmarkmetriken}, wie zum Beispiel die Anzahl der oben erwähnten "einfachen" Schleifen, können zur Beschreibung von Benchmarks herangezogen werden und derart die Vergleichbarkeit verbessern.

Motiviert vom Stand der Technik führt diese Diplomarbeit eine Studie von Schleifen in C Programmen durch, welche wir basierend auf der Schwierigkeit automatischer Programmanalyse klassifizieren: Wir führen die Klasse der \emph{syntaktisch terminierenden Schleifen} ein, die ein von Programmierern häufig benutztes Muster beschreiben und einen Terminierungsbeweis anhand minimaler Information erlauben. Wir leiten die Familie der \emph{simplen Schleifen} ab, indem wir die Kriterien syntaktisch terminierender Schleifen systematisch abschwächen. Dieser Ansatz deckt die Mehrzahl der Schleifen in unseren Benchmarks ab, während er gleichzeitig die Terminierung nach wie vor hinreichend gut beschreibt. Abschließend stellen wir unsere Implementierung \sloopy{} vor und vergleichen die identifizierten Klassen mit experimentellen Resultaten von \loopus{}, einem Tool zur Berechnung symbolischer Bounds von Schleifen. Derart zeigen wir, dass simple Schleifen tatsächlich die Schwierigkeit automatischer Programmanalyse erfassen.
